\chapter{Water Tank (Event-Driven)} \label{chap:watertankeventdriven}
\section{Case Description}
The event driven water tank model is very similar to the periodic
water tank model described in Chapter \ref{chap:watertankperiodic}.
It models a water tank that is supplied by a flow of water inwards.
There is a valve at the bottom of the tank which can be opened so that
water flows out of the tank, or closed to stop the outflowing water.
Underneath the tank we assume that there is a drain to catch any water
flowing out.  The tank is equipped with a sensor to measure the
current water level.

We should like the water in the tank to remain between some maximum
and some minimum level.  If the water level falls below our minimum
desired level then the valve should be closed to stop the outflow.  On
the other hand, if the water rises above our maximum desired level,
then the valve should be opened to enable water to flow out.

The event-driven model accomplishes this using events, which are
triggered on the CT side of the co-model.  When the water level sensor
detects that the water is too high, it triggers a appropriate event
and signals the DE model.  The DE model then triggers the valve to
open by signalling the valve actuator.  Similarly, if the sensor
detects a water level that is too low, the CT model triggers a
different event and signals the DE model; the DE model then closes the
valve.

\section{Contract}
The contract for the event-driven water tank model is very similar to
that for the periodic water tank model.  It includes one
\emph{monitored} variable of type \keyw{real} (called \texttt{level})
which is the signal representing the current level of water in the
tank as measured by the sensor.  And there is one \emph{controlled}
variable of type \keyw{bool} (called \texttt{valve}) which is a signal
produced by the DE model to open or shut the valve.

In addition, there are two shared design parameters of type
\keyw{real}: \texttt{maxlevel} and \texttt{minlevel}.  The shared
design parameters can be configured at runtime when executing a
simulation of the model, by opening the debug configuration options in
the DESTECS tool before the simulation is started.

Finally, the event-driven model also includes two declared events:
\texttt{high}, which is an event triggered when the water level passes
some maximum desired level; and \texttt{minLevelReached}, which is
triggered when the water level falls below some desirable minimum
level.

\section{Discrete-event} The DE model contains an
\texttt{Controller} class, which has the main thread of control.  An
instance of \texttt{LevelSensor} is created to represent the sensor
that measures the current water level, and also an instance of
\texttt{ValveActuator} is created to represent the valve at the bottom
of the tank.  An instance of \texttt{WatertankEventHandler} is created
to receive the events that will be triggered by the CT model.

\texttt{LevelSensor} returns a value to represent the current level of
water, whilst \texttt{ValveActuator} accepts a boolean value that sets
the valve to be open or closed.  We assume that the level sensor is
able to self-detect and report some errors, and the
\texttt{LevelSensor} class checks for these. The class
\texttt{WatertankEventHandler} exists to catch the events triggered by
the CT model and react to them.  It reacts to the \texttt{high} event
by opening the valve on the tank, and it reacts to the
\texttt{minLevelReached} by closing the valve.

\section{Continuous-time}
The CT model for the periodic water tank consists of two main parts,
one to represent the link to the DE side of the model (\texttt{DE\_link})
and one to represent the continuous-time side
(\texttt{I\_O\_Plant}).

The \texttt{discrete-event} block handles interaction with the DE
model itself.  It contains a block, \texttt{Control}, which handles
the input parameter produced by the DE model and also the parameter
for data output by the CT model.  It also houses some logic for
triggering the relevant events.  Events in this case are triggered by
a zero value being reached.  The event \texttt{minLevelReached} is
used to communicate to the DE model that the water level has fallen
below some minimum desirable level of water, and is calculated by
subtracting the minimum desired level from the actual current level; a
result at zero results in the triggering of the event
\texttt{minLevelReached}.  Similarly, the maximum desired level is
also subtracted from the current level; when the result is zero this
triggers the event \texttt{high}.

The \texttt{continuous-time} block incorporates other blocks to
represent the plant.  There is a \texttt{FlowSource} to represent the
source of incoming water, a \texttt{tank} to contain the water and a
\texttt{Valve} at the bottom of the tank to control outward flow.
Underneath the tank is a \texttt{Drain}.

\section{Usage}
As with the periodic water tank model, running a simulation of the
event-driven water tank will illustrate how the water level in the
tank rises and falls between the maximum and minimum levels.  One way
to experiment with the model is to change the shared design parameters
which dictate these levels and experimenting to observe the effect on
the model's behaviour.  These can be set in the Debug configuration
window in the DESTECS tool.
